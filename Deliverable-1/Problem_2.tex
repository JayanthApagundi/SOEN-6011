\documentclass[11pt,a4paper,twoside]{article}

\usepackage[dutch]{babel}
\usepackage[utf8]{inputenc}
\usepackage[T1]{fontenc}
\usepackage{amsmath,amssymb,amsfonts,textcomp}
\usepackage{enumitem}
\usepackage{geometry}

% Page Layout
\geometry{
  a4paper,
  top=2.5cm,
  bottom=2.5cm,
  left=3cm,
  right=3cm,
  headheight=15pt,
  headsep=0.3cm,
  footskip=1cm
}

\begin{document}

% Title Block
\begin{center}
{\Large \bfseries PROBLEM-2} \\[0.2cm]
{\bfseries Name:} Jayanth Apagundi \hfill {\bfseries ID:} 40291184 \\[0.1cm]
{\bfseries SOEN 6011: Software Engineering Processes} \hfill {\bfseries Date:} 13 July 2025
\end{center}

\section*{Requirements}

\begin{itemize}[leftmargin=0pt,label={}]
\item \textbf{- Requirement 1}
\begin{itemize}
\item \textbf{Identification:} FR1
\item \textbf{Version Number:} 1.0.0
\item \textbf{Owner:} Jayanth Apagundi
\item \textbf{Stakeholder Priority:} High
\item \textbf{Risk:} Low
\item \textbf{Description:} The program shall accept an angle input from the user as a real number.
\item \textbf{Rationale:} To enable computation of $\tan(x)$ based on user-provided angle value.
\item \textbf{Difficulty:} Easy
\item \textbf{Type:} Functional Requirement
\end{itemize}

\item \textbf{- Requirement 2}
\begin{itemize}
\item \textbf{Identification:} FR2
\item \textbf{Version Number:} 1.0.0
\item \textbf{Owner:} Jayanth Apagundi
\item \textbf{Stakeholder Priority:} High
\item \textbf{Risk:} Low
\item \textbf{Description:} The program shall allow the user to specify the unit of the input angle as degrees or radians by entering ‘d’ or ‘r’.
\item \textbf{Rationale:} To let the user input the angle in their preferred unit and handle conversions appropriately.
\item \textbf{Difficulty:} Easy
\item \textbf{Type:} Functional Requirement
\end{itemize}

\item \textbf{- Requirement 3}
\begin{itemize}
\item \textbf{Identification:} FR3
\item \textbf{Version Number:} 1.0.0
\item \textbf{Owner:} Jayanth Apagundi
\item \textbf{Stakeholder Priority:} Medium
\item \textbf{Risk:} Low
\item \textbf{Description:} The program shall convert the input angle from degrees to radians if the user specifies degrees.
\item \textbf{Rationale:} Trigonometric functions in Java work with radians; hence conversion is necessary.
\item \textbf{Difficulty:} Easy
\item \textbf{Type:} Functional Requirement
\end{itemize}

\item \textbf{- Requirement 4}
\begin{itemize}
\item \textbf{Identification:} FR4
\item \textbf{Version Number:} 1.0.0
\item \textbf{Owner:} Jayanth Apagundi
\item \textbf{Stakeholder Priority:} High
\item \textbf{Risk:} Medium
\item \textbf{Description:} The program shall compute $\tan(x)$ of the input angle and display the result.
\item \textbf{Rationale:} To provide the main functionality of calculating tangent.
\item \textbf{Difficulty:} Medium
\item \textbf{Type:} Functional Requirement
\end{itemize}

\item \textbf{- Requirement 5}
\begin{itemize}
\item \textbf{Identification:} FR5
\item \textbf{Version Number:} 1.0.0
\item \textbf{Owner:} Jayanth Apagundi
\item \textbf{Stakeholder Priority:} High
\item \textbf{Risk:} Medium
\item \textbf{Description:} The program shall detect and display an appropriate message if $\tan(x)$ is undefined.
\item \textbf{Rationale:} To avoid displaying invalid or infinite results to the user.
\item \textbf{Difficulty:} Medium
\item \textbf{Type:} Functional Requirement
\end{itemize}

\item \textbf{- Requirement 6}
\begin{itemize}
\item \textbf{Identification:} FR6
\item \textbf{Version Number:} 1.0.0
\item \textbf{Owner:} Jayanth Apagundi
\item \textbf{Stakeholder Priority:} Medium
\item \textbf{Risk:} Low
\item \textbf{Description:} The program shall display an error message if the unit specified is neither ‘d’ nor ‘r’.
\item \textbf{Rationale:} To guide the user to enter a valid unit and prevent incorrect behavior.
\item \textbf{Difficulty:} Easy
\item \textbf{Type:} Functional Requirement
\end{itemize}

\item \textbf{- Requirement 7}
\begin{itemize}
\item \textbf{Identification:} NFR1
\item \textbf{Version Number:} 1.0.0
\item \textbf{Owner:} Jayanth Apagundi
\item \textbf{Stakeholder Priority:} Medium
\item \textbf{Risk:} Low
\item \textbf{Description:} The program shall display the result of the calculation within 1 second after the user enters valid input.
\item \textbf{Rationale:} To ensure good performance and user satisfaction by providing timely results.
\item \textbf{Difficulty:} Easy
\item \textbf{Type:} Non-Functional Requirement
\end{itemize}

\item \textbf{- Requirement 8}
\begin{itemize}
\item \textbf{Identification:} NFR2
\item \textbf{Version Number:} 1.0.0
\item \textbf{Owner:} Jayanth Apagundi
\item \textbf{Stakeholder Priority:} High
\item \textbf{Risk:} Low
\item \textbf{Description:} The program shall provide clear and understandable prompts and error messages to guide the user through the input process.
\item \textbf{Rationale:} To improve usability and reduce user errors by making the interface user-friendly.
\item \textbf{Difficulty:} Easy
\item \textbf{Type:} Non-Functional Requirement
\end{itemize}

\end{itemize}

\section*{Assumptions}
\begin{itemize}
\item The user inputs a valid real number for the angle; no validation is done for non-numeric input.
\item The user specifies the unit correctly as either ‘d’ (degrees) or ‘r’ (radians); any other input is treated as invalid.
\item The calculator is operated in a console environment that supports standard input/output (e.g., terminal, command prompt).
\end{itemize}

\end{document}
